\section{Introduction}
In "\textit{Disquisitiones Arithmeticae}", Gauss stated, "Any composite number can be resolved into prime factors in only one way" (Shapiro, 2008, p.44) \cite{shapiro2008introduction}. That is, a composite number can be factored into a set of unique prime integers. Factorization of composite integers into their primes is considered as an interesting mathematical problem and has been an active area of research.\cite{unique_factorization} Although no classical algorithm has been developed to complete the task efficiently. A secure cryptographic technique called RSA(Rivest-Shamir-Adleman) cryptosystem was developed using the advantage of the inefficiency in solving this problem. RSA cryptosystem is used in securing our banking transactions, emails, and internet access communications.

We categorize a problem's efficiency by its complexity or the amount of resources consumed to solve the problem. The problem of integer factorization is considered NP(non-polynomial) problem classically. That is, it takes non-polynomial resources(time or space) to solve the problem. The best classical algorithm used for integer factorization called General Number Field Sieve has sub-exponential complexity. 

In this study, we discuss an algorithm for integer factorization called Shor's algorithm. It was developed by Peter Shor in 1994.\cite{shor1994} Shor's algorithm is a quantum algorithm, that is, it runs on a quantum computer which is inherently different from classical computers. The working mechanism of QCs is based on quantum mechanical properties like entanglement and superposition. This gives QCs a certain kind of advantages over classical ones. Shor's algorithm has polynomial time complexity with bounded error. That is, it performs better than the best classical algorithm for the task.

 Shor's algorithm takes an odd composite integer and factorizes it into its factors. It reduces the factorization problem to period finding problem. Use of quantum modular exponentiation and quantum Fourier transform cause exponential speed up in period finding.\cite{van2005fast} And finally, with the use of number theory principles, the obtained period can be used to determine a factor with some bounded probability.

The first experimental implementation of Shor's algorithm on a quantum computer was performed by a group at IBM in 2001 using nuclear magnetic resonance to factor an integer 15. \cite{vandersypen2001experimental} The largest integer ever factored using  Shor's algorithm is 21, performed in 2012 by scientists at university of Birstol.\cite{martin2012experimental}